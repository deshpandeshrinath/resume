%-------------------------
% Resume in Latex
% Author : Shrinath Deshpande
% License : MIT
%------------------------

\documentclass[letterpaper,10pt]{article}

\usepackage{latexsym}
\usepackage[empty]{fullpage}
\usepackage{titlesec}
\usepackage{marvosym}
\usepackage[usenames,dvipsnames]{color}
\usepackage{verbatim}
\usepackage{enumitem}
\usepackage[pdftex]{hyperref}
\usepackage{fancyhdr}


\hypersetup{
  colorlinks   = true, %Colours links instead of ugly boxes
  urlcolor     = blue, %Colour for external hyperlinks
  linkcolor    = blue, %Colour of internal links
  citecolor   = blue %Colour of citations
}

\pagestyle{fancy}
\fancyhf{} % clear all header and footer fields
\fancyfoot{}
\renewcommand{\headrulewidth}{0pt}
\renewcommand{\footrulewidth}{0pt}

% Adjust margins
\addtolength{\oddsidemargin}{-0.5in}
\addtolength{\evensidemargin}{-0.5in}
\addtolength{\textwidth}{1in}
\addtolength{\topmargin}{-.5in}
\addtolength{\textheight}{1.0in}

\urlstyle{same}

\raggedbottom
\raggedright
\setlength{\tabcolsep}{0in}

% Sections formatting
\titleformat{\section}{
  \vspace{-4pt}\scshape\raggedright\large
}{}{0em}{}[\color{black}\titlerule \vspace{-5pt}]

%-------------------------
% Custom commands
% Main Headings like Company name with Tabular structure of Company & Place // designation & Duration
\newcommand{\resumeHeading}[4]{
  \vspace{-1pt}
    \begin{tabular*}{0.97\textwidth}{l@{\extracolsep{\fill}}r}
      \textbf{#1} & #2 \vspace{-2pt}\\ \vspace{1pt}
      \textit{\small#3} & \textit{\small #4} \\
    \end{tabular*}
      %\vspace{-5pt}
}
% Sub Headings e.g. Project Titles
\newcommand{\resumeSubheading}[1]{
      {\small\textbf{#1}} \\
      %\vspace{-5pt}
}

\newcommand{\resumeSubheadingWithDate}[2]{
    \begin{tabular*}{0.97\textwidth}{l@{\extracolsep{\fill}}r}
      \small\textbf{#1} & \small #2 \\
    \end{tabular*}
    \vspace{+2pt}
}

\newcommand{\resumeSubheadingNew}[1]{
      {\small{#1}} \\
      %\vspace{-5pt}
}

% Reducing Gap Between Sections
\newcommand{\resumeSection}[1]{
\vspace{-12pt}
\section{\textbf{#1}}
}

% Bullet list for details
\newcommand{\resumeItemListStart}{
\vspace{-7pt}
\begin{itemize}[leftmargin=14pt]
}
\newcommand{\resumeItemListEnd}{
\vspace{+7pt}
\end{itemize}
}

\newcommand{\resumeItem}[1]{
  \item\small{
      {#1 \vspace{-7pt}
      }
  }
}

\renewcommand{\labelitemii}{$\circ$}


%-------------------------------------------
%%%%%%  CV STARTS HERE  %%%%%%%%%%%%%%%%%%%%%%%%%%%%


\begin{document}

%----------HEADING-----------------
\begin{tabular*}{\textwidth}{l@{\extracolsep{\fill}}c@{\extracolsep{\fill}}r}
 700, Health Sciences Drive, &\textbf{{\LARGE Shrinath Deshpande}} & Email : \href{mailto:deshpandeshrinath@gmail.com}{deshpandeshrinath@gmail.com}\\
 Stony Brook, New York  &\href{https://deshpandeshrinath.github.io/}{https://deshpandeshrinath.github.io/} & Mobile : +1-631-633-1851 \\
\end{tabular*}


%-----------EDUCATION-----------------
\section{\textbf{Education}}
    \resumeHeading
      {Stony Brook University}{Stony Brook, NY}
      {Ph.D. (Major: Mechanical, Minor: Computer Science), GPA 3.88}{Aug. 2015 -- Present}
    \resumeItemListStart
      %\resumeItem{\textbf{Relevent Coursework :} Advanced Control Systems, Robotics, Advanced Dynamics, Artificial Intelligence, Computer Vision, Machine Learning, Analysis of Algorithms, Computational Geometry, Geometric Modelling, Product Design Optimization}
      \resumeItem{\textbf{Relevent Coursework :} Artificial Intelligence, Computer Vision, Machine Learning, Analysis of Algorithms, Advanced Control Systems, Robotics, Advanced Dynamics, Computational Geometry, Geometric Modeling, Product Design Optimization}
      \resumeItem{Developing an ML-based framework for Mechanical Design, under the guidance of Dr. Purwar; funded by \$450K \href{https://nsf.gov/awardsearch/showAward?AWD_ID=1563413}{NSF grant.}}
    \resumeItemListEnd


%-----------EXPERIENCE-----------------
\resumeSection{Experience}
    \resumeHeading
      {Stony Brook University}{Stony Brook, NY}
      %{Research Assistant, Teaching Assistant}{May 2016 -- Present, Aug 2015 -- May 2016}
      {Research Assistant}{May 2016 -- Present}
      \resumeSubheading{Machine Learning}
      \resumeItemListStart
        \resumeItem{Interactive ML Agent for Assisted Mechanism Design; Human Machine Collaboration; ML for conceptual machine design.}
        \resumeItem{Developing Generative Models (VAE's and Conditional GAN's) for computational creativity and managing uncertainty.}
        \resumeItem{Working on deep RL model for Mechanical Design; Developed an OpenAI-gym Environment for Mechanism Design Task.}
        \resumeItem{VAE's for representation learning of Mechanism Motions in 2D and 3D; Trained classifiers for Mechanism Type-Synthesis}
      \resumeItemListEnd

      \vspace{-5pt}
      \resumeSubheading{Optimization}
      \resumeItemListStart
        \resumeItem{Developed Lagrange Optimization routine for four-bar linkage synthesis; Reduces constrained optimization into a polynomial system.
          Solved the system by gr\"{o}ebner basis method; implemented using GIAC npm package on node.js server.}
        \resumeItem{Lead Author of an award winning publication for solving practical synthesis problems (doi: 10.1115/1.4037801)}
      \resumeItemListEnd

      \vspace{-5pt}
      \resumeSubheading{MotionGen: Web, iOS and Android App for Linkage Synthesis}
      \resumeItemListStart
        \resumeItem{Developed smart-synthesis, motion interpolation functionalities for the cross platform app based on MVC architecture; \\ url: \href{http://cadcam.eng.sunysb.edu/}{http://cadcam.eng.sunysb.edu/}. Used Apache Cordova framework for iOS and Android implementations.}
        \resumeItem{Implemented multi-core computations for synthesis using \emph{Cluster} node package.}
      \resumeItemListEnd

      %\vspace{-5pt}
      %\resumeSubheadingNew{\textbf{Teaching Assistant} - MEC101 (Mechanical Design Innovation), MEC 262 Engineering Dynamics}
      %\resumeItemListStart
      %  \resumeItem{Involved in creating assignment, exams and conducting recitation sessions for 200\texttt{+} students in each course.}
      %  \resumeItem{Developed modular robotic kits for MEC101 students; Conducted Hands-On tutorials on Arduino programming.}
      %\resumeItemListEnd

%--------PROGRAMMING SKILLS------------
\resumeSection{Skills}
\vspace{+7pt}
    \resumeItemListStart
      \resumeItem{\textbf{Languages :} Proficient in Python, Javascript, MATLAB, Mathematica. Competent with C\texttt{++}, HTML5, CSS}
      \resumeItem{\textbf{Tools \& Technologies :} Tensorflow, OpenCV, Simulink, ROS, Vim, Scikit-learn, OpenGL, Canvas, Three.js, ReactJS}
    \resumeItemListEnd


%-----------PROJECTS-----------------
    \resumeSection{Relevant Projects}

    \resumeHeading{Deep Reinforcement Learning for Continuous Control Tasks}{CSE 537 AI, Prof. N Balasubramanian}{Tensorflow, OpenAI-Gym \href{https://github.com/deshpandeshrinath/deepDGP}{https://github.com/deshpandeshrinath/deepDGP}}{Jan 2018 -- May 2018}
    \resumeItemListStart
      \resumeItem{Implemented Deep DPG algorithm to learn continuous control policies; Compatible with all OpenAI-Gym environments.}
      \resumeItem{Implemented Hindsight Experience Replay for learning goal-oriented tasks with sparse binary rewards.}
    \resumeItemListEnd


    \resumeHeading{Visual Odometry with Deep Learning}{CSE527 Computer Vision, Prof. Roy Shilkrot} {Python, Tensorflow, OpenCV \href{https://github.com/sladebot/deepvo}{https://github.com/sladebot/deepvo}}{Oct 2017 -- Dec 2017}
    \resumeItemListStart
      \resumeItem{Built deep Recurrent Convolutional Neural Network for pose estimation of a car; CNN was derived from pretrained FlowNet2.0}
      \resumeItem{Trained and tested on KITTI visual odometry dataset (grayscale); Supported by \href{http://hi.cs.stonybrook.edu/}{Human Interaction Lab}, Stony Brook.}
    \resumeItemListEnd

    %%for face tracking intership at facebook, comment for others
    %    \resumeHeading{Face Tracking in a Video Sequence}{CSE527 Computer Vision, Prof. Roy Shilkrot} {Python, OpenCV}{Oct 2017 -- Nov 2017}
    %    \resumeItemListStart
    %      \resumeItem{Detected the face in first frame with Viola-Jones Detector; Compared the tracking by CAMShift, PF, KF,Optical Flow tracker.}
    %    \resumeItemListEnd
    %\vspace{-10pt}

    \resumeHeading{Computing Central Trajectory}{CSE555 Computational Geometry, Prof. Joseph Mitchell}{CGAL, OpenGL, Boost, C\texttt{++}}{March 2017 -- May 2017}
    \resumeItemListStart
      \resumeItem{Developed an algorithm to find valid representative trajectory among n time stamped trajectories; works in d dimensional space.}
      \resumeItem{Algorithm builds a weighted DAG on input; designed heuristics for assigning weights. Output is dijkstra's shortest path on DAG.}
    \resumeItemListEnd
    \vspace{-10pt}

    \resumeHeading{Optimal Control of a Drifting Car}{MEC560 Advanced Control Systems, Prof. Vivek Yadav}{MATLAB, GPOPS-II}{Oct 2016 -- Dec 2016}
    \resumeItemListStart
      \resumeItem{Designed Ext. Kalman Filter for Observer; Modeled governing dynamics; Used empirical tire friction model for drift simulations.}
      \resumeItem{Computed shortest path using Dynamic Programming. Obtained Optimal Control via Direct Collocation; Implemented in MATLAB using optimal control solver \href{http://www.gpops2.com/}{GPOPS II}.}
      \resumeItem{Used high gain PID controller to follow optimal control. Results match with empirical drifting techniques used by race drivers.}
    \resumeItemListEnd
    %\resumeHeading{Motion Planning of Baxter Arm}{MEC529 Robotics, Prof. N. Chakraborty}{MATLAB}{March 2016 -- May 2016}
    %\resumeItemListStart
    %  \resumeItem{Computed smooth B-Spline motion for pushing. Computed Jacobian matrix; Applied approximate Inverse Position Kinematics}
    %  \resumeItem{Obtained joint angles and rates for the task. Performed simulations to validate the results.}
    %\resumeItemListEnd

    %\resumeHeading{Wall Climbing Robot}{Senior Design Project}{MATLAB, Simulink, Solidworks}{Jan 2015 -- May 2015}
    %\resumeItemListStart
    %  \resumeItem{Designed a noval mechanism inspired by Gecko; mechanically synchronized gait; used vacuum technology for adhesion}
    %  \resumeItem{Developed robust feedback system; autonomous climbing; square shaped robot with area 1 $ft^2$ has climbing speed of 3 inch/sec}
    %\resumeItemListEnd
    \vspace{-10pt}
    \resumeHeading{Interactive Manipulation of NURBS Surfaces}{MEC572 Geomtric Modelling, Prof. Anurag Purwar}{C\texttt{++}, OpenGL}{March 2016 -- May 2016}
    \resumeItemListStart
      \resumeItem{QT5, OpenGL based implementation in C\texttt{++} for interactive manipulation of Non Uniform Rational B-Spline Surfaces.}
    \resumeItemListEnd

\resumeSection{Selected Publications}
\vspace{+7pt}
    \resumeItemListStart
      \resumeItem{Deshpande S, Purwar A. \textbf{Computational Creativity via Assisted Variational Synthesis of Mechanisms using Deep Generative Models},  ASME Journal of Mechanical Design 2019; Paper No: JMD-19-1152 (Accepted)}
      \resumeItem{Deshpande S, Purwar A. \textbf{A Machine Learning Approach to Kinematic Synthesis of Defect-Free Planar Four-Bar Linkages}, ASME IDETC Aug 2018. (Accepted for ASME J. Computing and Information Science in Engineering)}
      \resumeItem{Purwar, A., Deshpande, S., Ge, Q. J. \textbf{MotionGen: An iOS and Android App for Planar Four-Bar Motion Generation}, ASME IDETC 2016 August.}
      %\resumeItem{Deshpande, Shrinath, et al. "Wall-climbing robot with mechanically synchronized gait." Industrial Instrumentation and Control (ICIC), 2015 International Conference on. IEEE, 2015.}
    \resumeItemListEnd

\resumeSection{Awards}
    \resumeSubheadingWithDate{A.T. Yang Award in Theoretical Kinematics}{Aug 2017}
    \resumeItemListStart
      \resumeItem{Awarded \$1000 for the Best Paper at ASME Mechanisms and Robotics Conference, Cleveland, OH, August, 2017}
    \resumeItemListEnd

\end{document}
